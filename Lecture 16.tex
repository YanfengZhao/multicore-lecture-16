%
% This is the LaTeX template file for lecture notes for EE 382C/EE 361C.
%
% To familiarize yourself with this template, the body contains
% some examples of its use.  Look them over.  Then you can
% run LaTeX on this file.  After you have LaTeXed this file then
% you can look over the result either by printing it out with
% dvips or using xdvi.
%
% This template is based on the template for Prof. Sinclair's CS 270.

\documentclass[twoside]{article}
\usepackage{graphics}
\setlength{\oddsidemargin}{0.25 in}
\setlength{\evensidemargin}{-0.25 in}
\setlength{\topmargin}{-0.6 in}
\setlength{\textwidth}{6.5 in}
\setlength{\textheight}{8.5 in}
\setlength{\headsep}{0.75 in}
\setlength{\parindent}{0 in}
\setlength{\parskip}{0.1 in}

%
% The following commands set up the lecnum (lecture number)
% counter and make various numbering schemes work relative
% to the lecture number.
%
\newcounter{lecnum}
\renewcommand{\thepage}{\thelecnum-\arabic{page}}
\renewcommand{\thesection}{\thelecnum.\arabic{section}}
\renewcommand{\theequation}{\thelecnum.\arabic{equation}}
\renewcommand{\thefigure}{\thelecnum.\arabic{figure}}
\renewcommand{\thetable}{\thelecnum.\arabic{table}}

%
% The following macro is used to generate the header.
%
\newcommand{\lecture}[4]{
   \pagestyle{myheadings}
   \thispagestyle{plain}
   \newpage
   \setcounter{lecnum}{#1}
   \setcounter{page}{1}
   \noindent
   \begin{center}
   \framebox{
      \vbox{\vspace{2mm}
    \hbox to 6.28in { {\bf EE 382C/361C: Multicore Computing
                        \hfill Fall 2016} }
       \vspace{4mm}
       \hbox to 6.28in { {\Large \hfill Lecture #1: #2  \hfill} }
       \vspace{2mm}
       \hbox to 6.28in { {\it Lecturer: #3 \hfill Scribe: #4} }
      \vspace{2mm}}
   }
   \end{center}
   \markboth{Lecture #1: #2}{Lecture #1: #2}
   %{\bf Disclaimer}: {\it These notes have not been subjected to the
   %usual scrutiny reserved for formal publications.  They may be distributed
   %outside this class only with the permission of the Instructor.}
   \vspace*{4mm}
}

%
% Convention for citations is authors' initials followed by the year.
% For example, to cite a paper by Leighton and Maggs you would type
% \cite{LM89}, and to cite a paper by Strassen you would type \cite{S69}.
% (To avoid bibliography problems, for now we redefine the \cite command.)
% Also commands that create a suitable format for the reference list.
\renewcommand{\cite}[1]{[#1]}
\def\beginrefs{\begin{list}%
        {[\arabic{equation}]}{\usecounter{equation}
         \setlength{\leftmargin}{2.0truecm}\setlength{\labelsep}{0.4truecm}%
         \setlength{\labelwidth}{1.6truecm}}}
\def\endrefs{\end{list}}
\def\bibentry#1{\item[\hbox{[#1]}]}

%Use this command for a figure; it puts a figure in wherever you want it.
%usage: \fig{NUMBER}{SPACE-IN-INCHES}{CAPTION}
\newcommand{\fig}[3]{
			\vspace{#2}
			\begin{center}
			Figure \thelecnum.#1:~#3
			\end{center}
	}
% Use these for theorems, lemmas, proofs, etc.
\newtheorem{theorem}{Theorem}[lecnum]
\newtheorem{lemma}[theorem]{Lemma}
\newtheorem{proposition}[theorem]{Proposition}
\newtheorem{claim}[theorem]{Claim}
\newtheorem{corollary}[theorem]{Corollary}
\newtheorem{definition}[theorem]{Definition}
\newenvironment{proof}{{\bf Proof:}}{\hfill\rule{2mm}{2mm}}

% **** IF YOU WANT TO DEFINE ADDITIONAL MACROS FOR YOURSELF, PUT THEM HERE:

\begin{document}
%FILL IN THE RIGHT INFO.
%\lecture{**LECTURE-NUMBER**}{**DATE**}{**LECTURER**}{**SCRIBE**}
\lecture{16}{October 20}{Vijay Garg}{Yanfeng Zhao}
%\footnotetext{These notes are partially based on those of Nigel Mansell.}

% **** YOUR NOTES GO HERE:

% Some general latex examples and examples making use of the
% macros follow.  
%**** IN GENERAL, BE BRIEF. LONG SCRIBE NOTES, NO MATTER HOW WELL WRITTEN,
%**** ARE NEVER READ BY ANYBODY.
\section{Bitonic Sort}


\subsection{Monotonic Sequence}
\centerline{\includegraphics{Monotonic.png}}

\subsection{Bitonic Sequence}
\centerline{\includegraphics{BitonicSequence.png}}

{\bf Def}: $X_1 \dots X_n$ is a {\bf bitonic sequence} if there exist $k$ s.t. $X_1 \leq X_2 \leq \dots \leq X_k$ and $X_k \geq X_{k+1} \geq X_{k+2} \geq \dots \geq X_n$ or any circular rotation.

For example:

Bitonic sequence (k = 4): 1, 3, 5, 11, 9, 4, 2

Circular Shift Bitonic Sequence: 2, 1, 3, 5, 11, 9, 4. Notice that 2 was shifted to the front.

{\bf Any monotonic sequence is also bitonic sequence.}

\subsection{Why is this useful?}
\centerline{\includegraphics{whyUseful.png}}

{\bf Bitonic Merge}

$X_1 \dots X_{n/2} ->$ asc

$X_{n/2+1}\dots X_{n}$ -$>$ des

Max
\begin{itemize}
    \item $Y_1$ = max($X_1$, $X_{n/2+1}$)
    \item $Y_2$ = max($X_2$, $X_{n/2+2}$)
    \item ...
\end{itemize}
Min
\begin{itemize}
    \item $Z_1$ = max($X_1$, $X_{n/2+1}$)
    \item $Z_2$ = max($X_3$, $X_{n/2+3}$)
    \item ...
\end{itemize}

{\bf Bitonic Sort Example:}

2 7 11 13 9 5 3 1

2 5 3 1 $<$ 9 7 11 15

2 1 $<$ 3 5 $<$ 9 7 $<$ 11 13

1 $<$ 2 $<$ 3 $<$ 5 $<$ 7 $<$ 9 $<$ 11 $<$ 13

\section{Bitonic Sort Algorithm}
{\bf BitonicSort (lo, hi)} // assume size = $2^k$ for some $k$

    {\bf Base Case:} If array Size == 1, then return
    
    {\bf Recursion:}
\begin{enumerate}
	\item Bitonic Sort left side of the array (left half)
	\item Bitonic Sort right side of the array (right half)
	\item Compare and exchange if needed pairwise
	\item Bitonic Merge (L)
	\item Bitonic Merge (R)
\end{enumerate}

Steps 3-5 are part of the bitonic merge.

{\bf Example:}

\centerline{\includegraphics{bitonicSortBrick.png}}

\section{Bitonic Sort Time Complexity Analysis}
{\bf The algorithm in short:}

\begin{itemize}
    \item BitonicSort(L)
    \item BitonicSort(R)
    \item BitonicMerge(L, R)
\end{itemize}

Notice BitonicSort(L) and BitonicSort(R) can be done in parallel.

$T(n) = T(n/2) + log(n)$

$T(1) = O(1)$

$T(n) = T(n/2) + log(n)$
      
$= T(n/4) + log(n-1) + log(n)$
...
$= 1+2+3+...+log(n)$

$= O(log^2n)$



\section*{References}
\beginrefs
\bibentry{1}{\sc V. K. ~Garg},
 Introduction to Multicore Computing
\endrefs


\end{document}